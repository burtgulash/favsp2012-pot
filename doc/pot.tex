\documentclass[11pt]{article}
\usepackage[utf8]{inputenc}
\usepackage[czech]{babel}

\include{pygments}

\begin{document}
\begin{titlepage}
\begin{center}
	\mbox{} \\[3cm]
	\huge{Semestrální práce z předmětu KIV/POT} \\[2.5cm]
	\Large{Tomáš Maršálek} \\
	\large{marsalet@students.zcu.cz} \\[1cm]
	\normalsize{\today}
\end{center}
\thispagestyle{empty}
\end{titlepage}

\section{Zadání}
Dělení 16 bitů / 16 bitů = 16 bitů + 16 bitů (výsledek + zbytek) (bez použití
instrukce dělení). Vstupy a výstupy hexadecimálně.

\section{Řešení}
Na dělení je použit klasický algoritmus dělení (long division). Ve vyšším
jazyce odpovídá následujícímu kódu:
\begin{Verbatim}[commandchars=\\\{\}]
\PY{k+kt}{int} \PY{n+nf}{div}\PY{p}{(}\PY{k+kt}{int} \PY{n}{a}\PY{p}{,} \PY{k+kt}{int} \PY{n}{b}\PY{p}{)}
\PY{p}{\PYZob{}}
    \PY{k+kt}{int} \PY{n}{q}\PY{p}{,} \PY{n}{x}\PY{p}{;}

    \PY{n}{q} \PY{o}{=} \PY{l+m+mi}{0}\PY{p}{;}
    \PY{n}{x} \PY{o}{=} \PY{n}{b}\PY{p}{;}

    \PY{k}{while} \PY{p}{(}\PY{n}{x} \PY{o}{<}\PY{o}{=} \PY{p}{(}\PY{n}{a} \PY{o}{>}\PY{o}{>} \PY{l+m+mi}{1}\PY{p}{)}\PY{p}{)}
        \PY{n}{x} \PY{o}{<}\PY{o}{<}\PY{o}{=} \PY{l+m+mi}{1}\PY{p}{;}
    \PY{k}{while} \PY{p}{(}\PY{n}{x} \PY{o}{>}\PY{o}{=} \PY{n}{b}\PY{p}{)} \PY{p}{\PYZob{}}
        \PY{k}{if} \PY{p}{(}\PY{n}{a} \PY{o}{>}\PY{o}{=} \PY{n}{x}\PY{p}{)} \PY{p}{\PYZob{}}
            \PY{n}{q} \PY{o}{|}\PY{o}{=} \PY{l+m+mi}{1}\PY{p}{;}
            \PY{n}{a} \PY{o}{-}\PY{o}{=} \PY{n}{x}\PY{p}{;}
        \PY{p}{\PYZcb{}}
        \PY{n}{x} \PY{o}{>}\PY{o}{>}\PY{o}{=} \PY{l+m+mi}{1}\PY{p}{;}
        \PY{n}{q} \PY{o}{<}\PY{o}{<}\PY{o}{=} \PY{l+m+mi}{1}\PY{p}{;}
    \PY{p}{\PYZcb{}}
    \PY{n}{q} \PY{o}{>}\PY{o}{>}\PY{o}{=} \PY{l+m+mi}{1}\PY{p}{;}

    \PY{k}{return} \PY{n}{q}\PY{p}{;}
\PY{p}{\PYZcb{}}
\end{Verbatim}

Vstupy a výstupy používají pomocný buffer, ze kterého jsou Hornerovým schématem
rozkódovány, respektive zakódovány.

\section{Využitá paměť}
Program má v paměti uložené pomocné texty pro komunikaci s uživatelem, tzn. 
\begin{tabular}{|c|c|c|c|}
\hline
proměnná & typ & obsah & umístění v paměti \\
\hline
delenec  & .asciz & "Delenec:  " & \\
delitel  & .asciz & "Delitel:  " & \\
podil    & .asciz & "Podil:    " & \\
zbytek   & .asciz & "Zbytek:   " & \\
buffer   & .space & 32+1 \\
\hline
\end{tabular}

\end{document}
